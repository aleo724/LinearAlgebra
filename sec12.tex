\section{基底}
物理で力学を勉強すると,力の分解というのが非常に重要であることは承知できると思う.
ある力を表すベクトルを,2つの方向に分解する場合は,図のように平行四辺形を作ってやればいい.
そして,方向ごとに力の釣り合いなどを考えるのが力学の問題の典型的な解き方である.

もちろん,こうしたベクトルの分解の方法は1つではない.例えば,斜面を転がる台車の運動を考えるときは,通常の鉛直方向と水平方向への分解を考えるよりも,斜面方向とそれに垂直な方向への分解を考えたほうが簡単に運動方程式を立てることができるという経験はしたことがあると思う.
以下2次元のベクトルで,分解の方法が1つではないという例を示す.
\begin{example}
    $K^2$の任意のベクトル$\pmt{x_1 \\ x_2}$は,
    $\pmt{1 \\ 0}$というベクトルと$\pmt{0 \\ 1}$というベクトルを用いて,
    \[
        \pmt{x_1 \\ x_2} = x_1 \pmt{1 \\ 0} + x_2 \pmt{0 \\ 1}
    \]
    と書くことができる.
    しかし別の表現もできる.
    例えば,$\pmt{1 \\ 1}$と$\pmt{1 \\ -1}$を採用すれば,
    \[
        \pmt{x_1 \\ x_2} = \frac{x_1 + x_2}{2} \pmt{1 \\ 1} + \frac{x_1 - x_2}{2} \pmt{1 \\ -1}
    \]
    という分解もできる.
    しかし,$\pmt{1 \\ 0}$と$\pmt{2 \\ 0}$では,上のように表現できないベクトルもある.
    例えば,$\pmt{1 \\ 1}$は,第2成分が0でないため,
    どうやってもこの2つのベクトルのスカラー倍と和で書くことは不可能である.
    
    さらに,ベクトルの数を増やして,例えば$\pmt{1 \\ 0}$,$\pmt{0 \\ 1}$,$\pmt{1 \\ 1}$というベクトル3つで分解することを考えると,
    もちろんどんなベクトルも分解できるが,
    \[
        \pmt{2 \\ 1} = 2 \pmt{1 \\ 0} + \pmt{0 \\ 1} = \pmt{1 \\ 0} + \pmt{1 \\ 1}
    \]
    となるので,分解の仕方が複数存在することになる.
\end{example}
こうした,「分解」という考え方は,ベクトルだけでなく,他の場面にも有用である.例えば,通常の3項間漸化式は,等比数列を用いて分解することができる.
\begin{example}
    漸化式$a_{j+2} = 2a_{j+1} + 3a_j$を満たす数列$(a_j)$を考える.
    
    $a_j = \alpha^j$という形の等比数列で,
    上の漸化式を満たすものを探してみると,$\alpha$は$\alpha^2 = 2\alpha + 3$を満たす必要があることがわかる.よって,$\alpha = -1,3$であって,
    $((-1)^j)$と$(3^j)$という数列は上の漸化式を満たす.
    
    すると,この2つの数列をスカラー倍して足し合わせた$(\beta_1 (-1)^j + \beta_2 3^j)$という形の数列も,漸化式を満たしていることが直接計算で確認できる.
    
    さらに実は,上の漸化式を満たすような数列は,この形のものに限られるということも証明できる.
    実際,漸化式を変形すると,
    \[
        a_{j+2} + a_{j+1} = 3(a_{j+1} + a_j)
    \]
    となるので,$(a_{j+1} + a_j)$は公比3の等比数列であり,$a_{j+1} + a_j = b_2 3^j$と書けることがわかる.
    一方で,
    \[
        a_{j+2} - 3a_{j+1} = -(a_{j+1} - 3a_j)
    \]
    とも変形できるので,$(a_{j+1}-3a_j)$は公比$-1$の等比数列であり,$a_{j+1}-3a_j = b_1 (-1)^j$と書ける.よって,
    \[
        a_j = \frac{b_1}{4} (-1)^j + \frac{b_2}{4} 3^j
    \]
    となるので,結論を得る.
\end{example}
こうした,「和とスカラー倍で分解できるかどうか」ということを,
一般の$K$-ベクトル空間上で議論するための概念が次の\textbf{線形結合}と\textbf{基底}の考え方である.
\begin{dfn}
    $V$を$K$-ベクトル空間とする.
    \begin{enumerate}
        \item $v_1,v_2,\dots,v_n \in V$と$\alpha_1,\alpha_2,\dots,\alpha_n \in K$に対して,$\alpha_1 v_1 + \dots + \alpha_n v_n$という形の元を,$v_1,\dots,v_n$の\textbf{線形結合}(linear combination)という.
        \item $V$を$K$-ベクトル空間とし,$B \subset V$を部分集合とする.
        $B$が$V$上\textbf{線形独立}(linearly independent)\footnote{\textbf{一次独立}という言い方をすることもある.}であるとは,
        任意の有限個の$B$の元$v_1,\dots,v_n \in B$と,
        スカラー$\alpha_1,\dots,\alpha_n \in K$に対して,
        \[
            \alpha_1 v_1 + \dots + \alpha_n v_n = 0
        \]
        ならば,$\alpha_1 = \alpha_2 = \dots = \alpha_n = 0$となることである.
        \item $V$の部分集合$B \subset V$に対して,$B$が $V$を\textbf{生成する}(generate)とは,
        $0$でない任意の$V$の元が,$B$に属する有限個の元の線形結合として表現できることをいう.
        つまり,任意の$v \in V$に対して,
        ある有限個の$B$の元$v_1,\dots,v_m \in B$と,
        スカラー$\alpha_1,\dots,\alpha_m \in K$で,
        \[
            v = \alpha_1 v_1 + \dots + \alpha_m v_m
        \]
        となるものが存在することをいう.
        \item $V$の部分集合$B \subset V$が,$V$の\textbf{基底}(basis)であるとは,
        $B$が$V$を生成していて,なおかつ$B$が$V$上線形独立であることをいう.
    \end{enumerate}
\end{dfn}
「線型独立」「生成する」「基底」の概念を具体例で確認していこう.
\begin{example}\label{eg:standard_basis}
    各$1 \le j \le n$に対して,
    $e_j \in K^n$を,第$j$成分のみ1で,それ以外の成分は0であるようなベクトルとする.
    そして,
    \[
        B = \{e_1,e_2,\dots,e_n\}
    \]
    と置く.
    このとき,$B$は$K^n$の基底になる.
    
    線形独立であることを確認するには,
    \[
        \alpha_1 e_1 + \dots + \alpha_n e_n = 0
    \]
    となる$\alpha_1,\dots,\alpha_n \in K$はすべて0しかないということを示せばよい.
    実際,任意の$1 \le j \le n$に対して,上の式の左辺の第$j$成分は$\alpha_j$になる.
    従って,$\alpha_j = 0$が言える.
    
    また,生成することを確認するには,
    任意の$\pmt{x_1 \\ \vdots \\ x_n} \in K^n$が,$e_1,\dots,e_n$の線形結合で書けることを示せばいいが,
    これは,
    \[
        \pmt{x_1 \\ \vdots \\ x_n} = x_1 e_1 + \dots + x_n e_n
    \]
    となることから言える.
    
    このような基底を$K^n$の\textbf{標準基底}という.
\end{example}
\begin{example}
    まず,$V=K^3$として,
    \[
        B = \left\{ \pmt{1 \\ 2 \\ -1}, \pmt{2 \\ 3 \\ 1} \right\}
    \]
    という集合を考える.
    
    $B$は線形独立である.
    実際,
    \[
        \alpha_1 \pmt{1 \\ 2 \\ -1} + \alpha_2 \pmt{2 \\ 3 \\ 1} = \pmt{0 \\ 0 \\ 0}
    \]
    となったとする.
    すると,これを満たす$\alpha_1,\alpha_2 \in K$は,連立方程式
    \[
        \left\{
            \begin{array}{ll}
                 \alpha_1 + 2\alpha_2 &= 0  \\
                 2\alpha_1 + 3\alpha_2 &= 0 \\
                 -\alpha_1 + \alpha_2 &= 0
            \end{array}
        \right.
    \]
    の解である.
    これを解くと$\alpha_1 = \alpha_2 = 0$であるので,$B$は線形独立である.
    
    一方で,$B$は$V$を生成しない.
    実際,任意のベクトル$\pmt{x_1\\x_2\\x_3} \in K^3$が,
    \[
        \beta_1 \pmt{1 \\ 2 \\ -1} + \beta_2 \pmt{2 \\ 3 \\ 1} = \pmt{x_1 \\ x_2 \\ x_3}
    \]
    と表せるかどうかを考える.
    これは,$\beta_1,\beta_2$に対する連立方程式
    \begin{equation}\label{eq:eg-eq}
        \left\{
            \begin{array}{ll}
                 \beta_1 + 2\beta_2 &= x_1  \\
                 2\beta_1 + 3\beta_2 &= x_2 \\
                 -\beta_1 + \beta_2 &= x_3
            \end{array}
        \right.
    \end{equation}
    が解を持つかどうかを調べることに帰着する.
    この方程式の拡大係数行列は,
    \[
        \pmt{1 & 2 & x_1 \\ 2 & 3 & x_2 \\ -1 & 1 & x_3}
    \]
    であり,これを階段行列に変形すると,
    \[
        \pmt{1 & 0 & -3x_1 + 2x_2 \\ 0 & 1 & 2x_1 - x_2 \\ 0 & 0 & -5x_1 + 3x_2 + x_3}
    \]
    となる.
    したがって,連立方程式\eqref{eq:eg-eq}が解を持つための必要十分条件は,
    $-5x_1+3x_2+x_3=0$である.
    よって,$\pmt{x_1 \\ x_2 \\ x_3}$がこれを満たさなければ,
    \eqref{eq:eg-eq}は解を持たない.
\end{example}
\begin{example}
    同じく$V=K^3$上で,
    \[
        B = \left\{ \pmt{1 \\ 2 \\ -1}, \pmt{2 \\ 3 \\ 1}, \pmt{1 \\ 0 \\ -1}, \pmt{0 \\ 1 \\ 1}  \right\}
    \]
    という集合を考える.
    
    $B$は線形独立にはならない.実際
    \begin{equation}\label{eq:eg-rel}
        \alpha_1 \pmt{1 \\ 2 \\ -1} + \alpha_2 \pmt{2 \\ 3 \\ 1} + \alpha_3 \pmt{1 \\ 0 \\ -1} + \alpha_4 \pmt{0 \\ 1 \\ 1} = \pmt{0 \\ 0 \\ 0}
    \end{equation}
    となる$\alpha_1,\dots,\alpha_4 \in K$をとったとする.
    この方程式の係数行列は,
    \[
        \pmt{1 & 2 & 1 & 0 \\ 2 & 3 & 0 & 1 \\ -1 & 1 & -1 & 1}
    \]
    であり,これを階段行列に変形すると,
    \[
        \pmt{1 & 0 & 0 & 0 \\ 0 & 1 & 0 & \dfrac{1}{3} \\ 0 & 0 & 1 & -\dfrac{2}{3}}
    \]
    となるから,例えば$(\alpha_1,\alpha_2,\alpha_3,\alpha_4) = (0,-1,2,3)$とすれば,これは上の\eqref{eq:eg-rel}を満たしている.
    よって$B$は線形独立にはならない.
    
    一方,$B$は$K^3$を生成している.
    これは,任意の$\pmt{x_1 \\ x_2 \\ x_3} \in K^3$について,
    \begin{equation}\label{eq:eg-rel2}
        \beta_1 \pmt{1 \\ 2 \\ -1} + \beta_2 \pmt{2 \\ 3 \\ 1} + \beta_3 \pmt{1 \\ 0 \\ -1} + \beta_4 \pmt{0 \\ 1 \\ 1} = \pmt{x_1 \\ x_2 \\ x_3}
    \end{equation}
    となる$\beta_1,\dots,\beta_4$があることを示せばよい.
    拡大係数行列は,
    \[
        \pmt{1 & 2 & 1 & 0 & x_1 \\ 2 & 3 & 0 & 1 & x_2 \\ -1 & 1 & -1 & 1 & x_3}
    \]
    となり,これを階段行列に変形すると,
    \[
        \pmt{1 & 0 & 0 & 0 & \dfrac{x_1+x_2+x_3}{2} \\ 0 & 1 & 0 & \dfrac{1}{3} & \dfrac{x_1 + x_3}{3} \\ 0 & 0 & 1 & -\dfrac{2}{3} & \dfrac{5x_1 - 3x_2 - x_3}{6}}
    \]
    となる.拡大係数行列と係数行列の階数が一致するので,\eqref{eq:eg-rel2}は必ず解を持つことがわかる.
\end{example}
\begin{example}
    $n$を正の整数として,
    $K_n[x]$を,$n$次以下の多項式全体のなす$K$-ベクトル空間とする.つまり,
    \[
        K_n[x] := \{ \alpha_n x^n + \dots + \alpha_1 x_1 + \alpha_0\,;\, \alpha_0,\alpha_1,\dots,\alpha_n \in K \}
    \]
    とする.
    このとき,集合
    \[
        B = \{1,x,\dots,x^n\}
    \]
    は$K_n[x]$の基底になる.
    実際,$B$が線形独立であることは,
    \[
        \alpha_n x^n + \dots + \alpha_1 x_1 + \alpha_0 = 0
    \]
    が多項式として成り立つような$\alpha_0,\dots,\alpha_n$が自明なものしかないことから言える.
    また,$B$が$K_n[x]$を生成することは,$K_n[x]$の定義から明らかである.
\end{example}

次の命題は当たり前に思えるかもしれないが,概念のイメージを掴むためには述べておくべきものである.
\begin{prop}\label{prop:subset_indep_gen}
\begin{enumerate}
    \item $B\subset V$が線型独立であるとする.
    このとき,$B$の任意の部分集合$B' \subset B$も線型独立である.
    \item $B \subset V$が$V$を生成するとする.
    このとき,$B$を部分集合として含む任意の集合$B' \supset B$も$V$を生成する.
\end{enumerate}
\end{prop}
\begin{exercise}
上の\cref{prop:subset_indep_gen}を証明せよ.
\end{exercise}
\cref{prop:subset_indep_gen}が言おうとしていることは,直観的には次のようなことである:
\begin{itemize}
    \item ベクトルの個数が少なければ少ないほど,線型独立にはなりやすい.
    \item ベクトルの個数が多ければ多いほど,$V$を生成しやすい.
\end{itemize}
このように,「線型独立」と「生成」はある意味真逆の性質を持っている.だから,「線型独立」と「生成」の両方を要求する「基底である」という性質は,この2つの性質のバランスが取れているということを意味している.

このことを以下の定理で踏み込んで示しておく.
\begin{thm}\label{thm:basis_properties}
    $V$を$K$-ベクトル空間とし,$B \subset V$を$0$を含まない部分集合とする.
    このとき,以下はすべて同値である.
    \begin{enumerate}
        \item $B$は$V$の基底である.
        \item 任意の0でない$v \in V$に対して,
        $v$は$B$の有限個の元の線形結合として表すことができ,
        さらにそれは1通りに定まる.
        \item $B$は$V$を生成し,さらに$B$の真部分集合で$V$を生成するものは存在しない.
        \item $B$は$V$上線形独立で,$B$を含む$V$の部分集合で線形独立になるようなものは存在しない.
    \end{enumerate}
\end{thm}
\begin{proof}
    $(1) \Rightarrow (2)$:
    仮定より$B$は$V$の基底なので,特に$B$は$V$を生成する.
    従って,任意の$0 \neq v \in V$は$B$の有限個の元の線形結合として表すことができる.
    次にその表し方が1通りしかないことを示す.今,$v$が$B$の元による線形結合で以下のように2つに表せたとしよう.
    \[
        \begin{aligned}
            v &= \alpha_1 v_1 + \dots + \alpha_n v_n, \\
            v &= \alpha'_1 v'_1 + \dots + \alpha'_{n'} v'_{n'}.
        \end{aligned}
    \]
    ここで,$v_1,\dots,v_n$と$v'_1,\dots,v'_{n'}$の中には,共通する元があるかもしれない.
    今$m$個のベクトルが共通していたとすると,番号付けを適切に変えることによって,
    \[
        \begin{aligned}
            v &= \alpha_1 v_1 + \dots + \alpha_m v_m + \alpha_{m+1} v_{m+1} + \dots + \alpha_n v_n, \\
            v &= \alpha'_1 v_1 + \dots + \alpha'_m v_m + \alpha'_{m+1} v'_{m+1} + \dots + \alpha'_{n'} v'_{n'}.
        \end{aligned}
    \]
    と書けていると考えてよい.
    このとき,両辺の差を取って整理すると,
    \[
        (\alpha_1-\alpha'_1) v_1 + \dots + (\alpha_m - \alpha'_m) v_m + \alpha_{m+1} v_{m+1} + \dots + \alpha_n v_n  + (-\alpha'_{m+1}) v'_{m+1} + \dots + (- \alpha'_{n'}) v'_{n'} = 0
    \]
    となる.$v_1,\dots,v_n,v'_{m+1},\dots,v'_{n'}$は$B$の元であるので,特に線形独立でもある.従って,上の式の左辺に出てくるスカラーはすべて0にならなければならない.従って,
    \[
        \begin{aligned}
            \alpha_j &= \alpha'_j, &(1 \le j \le m) \\
            \alpha_k &= 0, &(m+1 \le k \le n) \\
            \alpha'_{\ell} &= 0, &(m+1 \le \ell \le n')
        \end{aligned}
    \]
    が成り立つ.従って,$v$の表示は1通りに定まることがわかった.
    
    $(2) \Rightarrow (1)$: (2)の仮定から,$B$が$V$を生成することは明らかなので,
    $B$が$V$上線形独立であることを示せば十分である.
    有限個の元$v_1,\dots,v_n \in B$と,スカラー$\alpha_1,\dots,\alpha_n \in K$で,
    \[
        \alpha_1 v_1 + \dots + \alpha_n v_n = 0
    \]
    となるものが存在したと仮定する.
    もし,$\alpha_1,\dots,\alpha_n$の中に0でないものが存在したとする.
    必要であれば番号をつけ直すことで,$\alpha_1 \neq 0$と仮定しても構わない.
    すると,
    \[
        v_1 = -\frac{\alpha_2}{\alpha_1} v_2 - \dots - \frac{\alpha_n}{\alpha_1} v_n
    \]
    である.
    ここで,$v_1,\dots,v_n \in B$であるので,この式の両辺はどちらも$B$の元による線形結合である.
    しかしこれは,同じ元を$B$の元の線形結合として表す方法が2通りあることになり,(2)の仮定と矛盾する.以上で$B$の線形独立性がわかる.
    
    $(1) \Rightarrow (3)$: $B$が$V$を生成することは基底の定義から明らかである.
    もし,$B$の真部分集合$B' \subset B$で,$B'$が$V$を生成するものが存在したと仮定する.
    そこで,$B$に含まれるが$B'$には含まれない元$v$を取る.
    仮定から,これは$B'$の元の線形結合で書けるので,
    有限個の元$v'_1,\dots,v'_n \in B'$と,スカラー$\alpha_1,\dots,\alpha_n \in K$で,
    \[
        v = \alpha_1 v'_1 + \dots + \alpha_n v'_n
    \]
    となるものが存在する.左辺を右辺に移項して,
    \[
        -v + \alpha_1 v'_1 + \dots + \alpha_n v'_n = 0
    \]
    を得る.
    ところで,$B' \subset B$なので,$v,v'_1,\dots,v'_n$はすべて$B$の元である.
    $B$は基底なので,特に線形独立でもある.
    なので,$v,v'_1,\dots,v'_n$の線形結合で0が作れたとすれば,すべてのベクトルの係数は0にならなければいけない.
    しかし,上の式で$v$の係数は$-1 \neq 0$なので,矛盾する.
    以上で結論を得る.
    
    $(3) \Rightarrow (1)$: これも$B$が線形独立であることを証明すればよい.
    有限個の元$v_1,\dots,v_n \in B$と,スカラー$\alpha_1,\dots,\alpha_n \in K$で,
    \[
        \alpha_1 v_1 + \dots + \alpha_n v_n = 0
    \]
    となるものが存在したと仮定する.
    先程と同じく,$\alpha_1,\dots,\alpha_n$の中に0でないものが存在したとして,
    必要であれば番号をつけ直して,$\alpha_1 \neq 0$と仮定しておく.
    すると,
    \begin{equation}\label{eq:v1_expression}
        v_1 = -\frac{\alpha_2}{\alpha_1} v_2 - \dots - \frac{\alpha_n}{\alpha_1} v_n
    \end{equation}
    であることは先程見たとおりである.
    すると,$B$から$v_1$を除いた集合$B \setminus \{v_1\}$も$V$を生成する.
    実際,もしも,ある$v \in V$が,$v_1$と他の元$v'_2,\dots,v'_m \in B$によって,
    \[
        v = \beta_1 v_1 + \beta_2 v'_2 + \dots + \beta_m v'_m
    \]
    と書けたとする.
    このとき,\eqref{eq:v1_expression}を代入すると,
    \[
        v = -\frac{\alpha_2\beta_1}{\alpha_1} v_2 - \dots - \frac{\alpha_n\beta_1}{\alpha_1} v_n + \beta_2 v'_2 + \dots + \beta_m v'_m
    \]
    となり,$v$は$v_1$以外の$B$の元の線形結合として表せることがわかる.
    しかし,$B \setminus \{v_1\}$は$B$の真部分集合なので,この事実は(3)に反する.
    以上で結論を得る.
    
    $(1) \Rightarrow (4)$:
    $B$を含む$V$の部分集合$B \subset B'$を任意に取る.
    すると,$v \in B' \setminus B$が取れる.$B$は基底なので,$v$が$B$の元の線形結合として,
    \[
        \alpha_1 v_1 + \dots + \alpha_n v_n = v
    \]
    と書ける.
    すると,
    \[
        \alpha_1 v_1 + \dots + \alpha_n v_n - v = 0
    \]
    であるが,この式の左辺は$B'$の元の線形結合である.
    $v$の係数が0でないので,ないことがわかる.$B'$は線形独立ではないことがわかる.
    
    $(4)\Rightarrow (1)$:
    $B$が$V$を生成することを示せばよい.
    もし,$V$の元$v$で,$B$の元の線形結合で書けないものが存在したと仮定する.
    このとき,$B \cup \{v\}$が線形独立であることを示す.
    $B \cup \{v\}$の有限個の元$v_1,\dots,v_n$とスカラー$\alpha_1,\dots,\alpha_n$で,
    \[
        \alpha_1 v_1 + \dots + \alpha_n v_n = 0
    \]
    となるものが存在したとする.
    もし$v_1,\dots,v_n$がすべて$B$の元であるなら,$B$の線形独立性より,$\alpha_1 = \dots = \alpha_n = 0$である.
    そこで,$v_1,\dots,v_n$のどれかが$v$であると仮定する.
    必要であれば番号をつけかえることで,$v_1 = v$としてもよい.
    このとき,$\alpha_1 = 0$だとすると,残りの$v_2,\dots,v_n$はすべて$B$の元だから,
    再び$B$の線形独立性から$\alpha_1 = \dots = \alpha_n = 0$が言える.
    なので,$\alpha_1 \neq 0$と仮定してよい.このとき,
    \[
        v = -\frac{\alpha_2}{\alpha_1} v_2 - \dots - \frac{\alpha_n}{\alpha_1} v_n
    \]
    となるが,これは$v$が$B$の元$v_2,\dots,v_n$の線形結合で表せるということを意味し,$v$が$B$の元の線形結合で書けないという仮定に反する.
    よって,$B \cup \{v\}$は線形独立であるが,これは(4)の仮定に反している.
    以上で結論を得る.
\end{proof}
どんなベクトル空間でも基底が存在するが知られているが,その証明には予備知識が必要となる.
\begin{thm}
    任意の$K$-ベクトル空間には基底が存在する.
\end{thm}